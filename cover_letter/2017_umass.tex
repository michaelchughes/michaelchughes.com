%% Modified from original file `template.tex'.
%% Copyright 2006-2013 Xavier Danaux (xdanaux@gmail.com).
%
% This work may be distributed and/or modified under the
% conditions of the LaTeX Project Public License version 1.3c,
% available at http://www.latex-project.org/lppl/.


\documentclass[11pt,letterpaper,roman]{moderncv}        % possible options include font size ('10pt', '11pt' and '12pt'), paper size ('a4paper', 'letterpaper', 'a5paper', 'legalpaper', 'executivepaper' and 'landscape') and font family ('sans' and 'roman')

% moderncv themes
\moderncvstyle{classic}                            % style options are 'casual' (default), 'classic', 'oldstyle' and 'banking'
\moderncvcolor{grey}                              % color options 'blue' (default), 'orange', 'green', 'red', 'purple', 'grey' and 'black'
%\renewcommand{\familydefault}{\sfdefault}         % to set the default font; use '\sfdefault' for the default sans serif font, '\rmdefault' for the default roman one, or any tex font name
%\nopagenumbers{}                                  % uncomment to suppress automatic page numbering for CVs longer than one page

% character encoding
\usepackage[utf8]{inputenc}

% adjust the page margins
\usepackage[letterpaper,scale={0.75,0.8}]{geometry}

% font
\usepackage{tgpagella}
\renewcommand{\familydefault}{\sfdefault}         

\renewcommand*{\addresssymbol}       {}
\renewcommand*{\mobilephonesymbol}   {}
\renewcommand*{\emailsymbol}         {}
\renewcommand*{\homepagesymbol}      {}
\renewcommand*{\addressfont}{\small\mdseries}

% personal data
\name{Michael C.}{Hughes}
\phone[mobile]{+1~(617)~371~7611}
\email{mike@michaelchughes.com}
\homepage{www.michaelchughes.com}
%\extrainfo{additional information}                 % optional, remove / comment the line if not wanted
%\photo[64pt][0.4pt]{picture}                       % optional, remove / comment the line if not wanted; '64pt' is the height the picture must be resized to, 0.4pt is the thickness of the frame around it (put it to 0pt for no frame) and 'picture' is the name of the picture file

%----------------------------------------------------------------------------------
%            content
%----------------------------------------------------------------------------------
\begin{document}
%-----       letter       ---------------------------------------------------------
% recipient data
\recipient{U. Mass.-Amherst Faculty Search Committee}{College of Information and Computer Sciences \\ 140 Governors Dr. \\ Amherst, MA 01003}
\date{\today}
\opening{Dear Members of the Faculty Search Committee,}
\closing{Sincerely,}

\makelettertitle

I am writing to apply for a tenure-track assistant professorship in computer science at U. Mass-Amherst. I believe my background in statistical machine learning and its applications to healthcare make me a good fit for the advertised data science position within the College of Information and Computer Sciences.

I am currently a postdoctoral fellow at Harvard's School of Engineering and Applied Sciences, after completing my Ph.D. in Computer Science from Brown University.
I develop machine learning methods
that discover actionable knowledge from large, messy datasets. 
My focus is healthcare applications, where new probabilistic models and optimization algorithms are needed to learn from abundant yet noisy and unlabeled electronic health records to produce insights clinicians can trust.
In one project, we modeled common trajectories in the Intensive Care Unit (ICU) to predict hourly need for interventions like mechanical ventilation. In ongoing work with Massachusetts General Hospital (MGH), we are discovering subpopulations of major depression disorder that require distinct drug combination therapies. Our efforts build on the core contributions of my Ph.D. thesis: reliable algorithms for clustering hierarchical or sequential datasets which can scale up to cover millions of articles or the whole human genome.
I strive to deliver results beyond top-tier publications, such as my open-source BNPy toolbox (used by data scientists at the NY Times). I hope to eventually deploy clinical recommendation systems integrated at the point-of-care.

Looking forward, I am excited to pursue
a broad data science agenda that delivers interpretable, reliable, and meaningful solutions in healthcare and other applications.
At U. Mass Amherst, I am excited about collaboration opportunities with experts in artificial intelligence, natural language processing, and bioinformatics to tame large-scale multimodal health datasets. 
I am particularly excited by many initiatives under the Computational Social Science Institute and the Center for Data Science, such as the Mobile Data to Knowledge center. I am eager to join the U. Mass. research community, and I look forward to hearing from you.

\makeletterclosing

\end{document}


%% end of file `template.tex'.
