%% start of file `template.tex'.
%% Copyright 2006-2013 Xavier Danaux (xdanaux@gmail.com).
%
% This work may be distributed and/or modified under the
% conditions of the LaTeX Project Public License version 1.3c,
% available at http://www.latex-project.org/lppl/.


\documentclass[11pt,letterpaper,roman]{moderncv}        % possible options include font size ('10pt', '11pt' and '12pt'), paper size ('a4paper', 'letterpaper', 'a5paper', 'legalpaper', 'executivepaper' and 'landscape') and font family ('sans' and 'roman')

% moderncv themes
\moderncvstyle{classic}                            % style options are 'casual' (default), 'classic', 'oldstyle' and 'banking'
\moderncvcolor{grey}                              % color options 'blue' (default), 'orange', 'green', 'red', 'purple', 'grey' and 'black'
%\renewcommand{\familydefault}{\sfdefault}         % to set the default font; use '\sfdefault' for the default sans serif font, '\rmdefault' for the default roman one, or any tex font name
%\nopagenumbers{}                                  % uncomment to suppress automatic page numbering for CVs longer than one page

% character encoding
\usepackage[utf8]{inputenc}

% adjust the page margins
\usepackage[scale=0.75]{geometry}

% font
\usepackage{tgpagella}
\renewcommand{\familydefault}{\sfdefault}         

\renewcommand*{\addresssymbol}       {}
\renewcommand*{\mobilephonesymbol}   {}
\renewcommand*{\emailsymbol}         {}
\renewcommand*{\homepagesymbol}      {}
\renewcommand*{\addressfont}{\small\mdseries}

% personal data
\name{Michael C.}{Hughes}
\phone[mobile]{+1~(617)~371~7611}
\email{mike@michaelchughes.com}
\homepage{www.michaelchughes.com}
%\extrainfo{additional information}                 % optional, remove / comment the line if not wanted
%\photo[64pt][0.4pt]{picture}                       % optional, remove / comment the line if not wanted; '64pt' is the height the picture must be resized to, 0.4pt is the thickness of the frame around it (put it to 0pt for no frame) and 'picture' is the name of the picture file

%----------------------------------------------------------------------------------
%            content
%----------------------------------------------------------------------------------
\begin{document}
%-----       letter       ---------------------------------------------------------
% recipient data
\recipient{Olin College, Faculty Search Committee\\ Needham, MA}{}
\date{October 28, 2017}
\opening{Dear Members of the Faculty Search Committee,}
\closing{Sincerely,}
%\enclosure[Attached]{curriculum vit\ae{}}          % use an optional argument to use a string other than "Enclosure", or redefine \enclname
\makelettertitle

I am writing to apply for a faculty position in computing.
I am currently a postdoctoral researcher at Harvard's School of Engineering and Applied Sciences (SEAS), after completing my Ph.D. in Computer Science from Brown University.
I develop machine learning methods 
to find useful structure in large, messy datasets.
My focus is healthcare applications, where novel probabilistic models and optimization algorithms are needed to learn from abundant yet noisy and unlabeled data to produce recommendations clinicians can trust.
In one project, we modeled common patient trajectories in the Intensive Care Unit (ICU) to suggest personalized interventions. In ongoing work with psychiatrists at Massachusetts General Hospital (MGH), we are discovering subpopulations of patients with major depression which require different drug combination therapies. 
I strive to deliver results beyond top-tier publications, including open-source software and (eventually) recommendation systems integrated at the point-of-care. 

As an independent researcher and educator, I will advance a broad data science agenda that delivers \emph{interpretable}, \emph{reliable}, and \emph{meaningful} solutions. Given my own Olin education (Class of 2010), I am passionate about a \textbf{human-oriented} approach to healthcare informatics. I have built a network of clinical collaborators at both MGH and Beth Israel Deaconess Medical Center to consult weekly about proposed solutions. I have shadowed ICU nurses on rounds to understand how they turn sensor data into treatment decisions. Producing useful health IT solutions requires a blend of technical, ethical, and political awareness. 
Olin graduates are thus uniquely positioned to be leaders in the future of data science, and I'd like join the Olin faculty to lead the way.

As a professor at Olin, I would engage in and outside the classroom. I could teach many hands-on computing and applied math courses. I would happily co-develop courses exploring the intersections of data science and X, where X could be human-computer interaction, biology, or the humanities. I can imagine a course that teams Olin students with medical students to codesign new patient-oriented tools.
I could also advise student entrepreneurship, 
since I am a scientific adviser for a well-funded AI stealth startup. Finally, I am passionate about outreach efforts that would export the Olin mindset to undergraduate and K-12 education programs, especially those that reach underserved communities. I am eager to joint the Olin community again as we work together toward our common goals.

\makeletterclosing

\end{document}


%% end of file `template.tex'.
