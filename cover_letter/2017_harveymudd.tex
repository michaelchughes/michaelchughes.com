%% start of file `template.tex'.
%% Copyright 2006-2013 Xavier Danaux (xdanaux@gmail.com).
%
% This work may be distributed and/or modified under the
% conditions of the LaTeX Project Public License version 1.3c,
% available at http://www.latex-project.org/lppl/.


\documentclass[11pt,letterpaper,roman]{moderncv}        % possible options include font size ('10pt', '11pt' and '12pt'), paper size ('a4paper', 'letterpaper', 'a5paper', 'legalpaper', 'executivepaper' and 'landscape') and font family ('sans' and 'roman')

% moderncv themes
\moderncvstyle{classic}                            % style options are 'casual' (default), 'classic', 'oldstyle' and 'banking'
\moderncvcolor{grey}                              % color options 'blue' (default), 'orange', 'green', 'red', 'purple', 'grey' and 'black'
%\renewcommand{\familydefault}{\sfdefault}         % to set the default font; use '\sfdefault' for the default sans serif font, '\rmdefault' for the default roman one, or any tex font name
%\nopagenumbers{}                                  % uncomment to suppress automatic page numbering for CVs longer than one page

% character encoding
\usepackage[utf8]{inputenc}

% adjust the page margins
\usepackage[letterpaper,scale={0.75,0.8}]{geometry}

% font
\usepackage{tgpagella}
\renewcommand{\familydefault}{\sfdefault}         

\renewcommand*{\addresssymbol}       {}
\renewcommand*{\mobilephonesymbol}   {}
\renewcommand*{\emailsymbol}         {}
\renewcommand*{\homepagesymbol}      {}
\renewcommand*{\addressfont}{\small\mdseries}

% personal data
\name{Michael C.}{Hughes}
\phone[mobile]{+1~(617)~371~7611}
\email{mike@michaelchughes.com}
\homepage{www.michaelchughes.com}
%\extrainfo{additional information}                 % optional, remove / comment the line if not wanted
%\photo[64pt][0.4pt]{picture}                       % optional, remove / comment the line if not wanted; '64pt' is the height the picture must be resized to, 0.4pt is the thickness of the frame around it (put it to 0pt for no frame) and 'picture' is the name of the picture file

%----------------------------------------------------------------------------------
%            content
%----------------------------------------------------------------------------------
\begin{document}
%-----       letter       ---------------------------------------------------------
% recipient data
\recipient{Olin College Faculty Search Committee}{}
\date{October 28, 2017}
\opening{Dear Members of the Faculty Search Committee,}
\closing{Sincerely,}
%\enclosure[Attached]{curriculum vit\ae{}}          % use an optional argument to use a string other than "Enclosure", or redefine \enclname
\makelettertitle

I am writing to apply for a faculty position in computing.
I am currently a postdoctoral fellow at Harvard's School of Engineering and Applied Sciences (SEAS), after completing my Ph.D. in Computer Science from Brown University.
I develop machine learning methods
that discover actionable knowledge from large, messy datasets. 
My focus is healthcare applications, where new probabilistic models and optimization algorithms are needed to learn from abundant yet noisy and unlabeled data to produce insights clinicians can trust.
In one project, we modeled common trajectories in the Intensive Care Unit (ICU) to predict hourly need for interventions like mechanical ventilation. In ongoing work with Massachusetts General Hospital (MGH), we are discovering subpopulations of major depression disorder that require distinct drug combination therapies. Our efforts build on the core contributions of my Ph.D. thesis: reliable algorithms for clustering hierarchical or sequential datasets.
I strive to deliver results beyond top-tier publications, such as my open-source BNPy toolbox (used by data scientists at the NY Times). I hope to eventually deploy clinical recommendation systems integrated at the point-of-care.

%As an independent researcher and educator, I will advance a broad data science agenda that delivers interpretable, reliable, and meaningful solutions.
Given my own Olin education (Class of 2010), I am passionate about a \emph{human-oriented} approach to healthcare informatics. I have built a network of collaborators at both MGH and Beth Israel Deaconess Medical Center to consult weekly about proposed solutions. I have shadowed ICU nurses on rounds to understand how they turn sensor data into treatment decisions. Such conversations lead to our ICU project's focus on predicting the need for mechanical ventilation \emph{several hours} in advance, leaving more time for preparation while the patient can talk freely and improving logistics for the nursing staff.
Machine learning can do much to improve healthcare, but it will require Olin's holistic blend of technical, ethical, and political awareness. 
I'd like to join the Olin faculty to help lead the way toward interpretable, reliable, and meaningful solutions.
%Olin graduates are thus positioned to be leaders in the future of data science, and I'd like join the Olin faculty to lead the way.

As a professor at Olin, I could teach many hands-on computing and applied math courses that emphasize quantitative rigor and effective communication. I would happily co-develop courses at the intersection of data science and X, where X could be human-computer interaction, biology, the humanities, and beyond. I can imagine courses where Oliners and medical students codesign new patient-oriented tools.
I could also mentor entrepreneurship, 
since I am a scientific adviser for a well-funded AI stealth startup. I have significant experience in STEM outreach (cofounder of Olin Engineering Discovery, Microsoft TEALS instructor), and I will continue to export the Olin mindset to other undergraduate and K-12 institutions, especially to underrepresented groups.
I am eager to join the Olin community again and work together toward our common goals.


\makeletterclosing

\end{document}


%% end of file `template.tex'.
